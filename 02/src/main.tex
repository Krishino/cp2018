\documentclass{scrartcl}
\usepackage{graphicx}
\usepackage{amsmath}
\usepackage{subcaption}
\usepackage[
  section, % Floats innerhalb der Section halten
  below,   % unterhalb der Section aber auf der selben Seite ist ok
]{placeins}

\title{Übungsblatt 02}
\author{%
  Noah Biederbeck, Maximilian Sackel, Jan Spinne
}
\date{Abgabe: 04. Mai 2018}



\begin{document}
\maketitle

\section*{Aufgabe 1}
\subsection*{b)}
Es werden Zufallszahlen nach der Vorschrift $r_{n+1} = (a \cdot r_n + c)\% m$ generiert.
Diese werden anschliessend durch $m$ geteilt, um gleichverteilte Zahlen in $\left[0, 1\right[$ zu erhalten.
Für vier Parametersätze wird der Generator getestet, die gleichverteilten Zufallszahlen sind in Abbildung~1 aufgetragen.
Man erkennt eine gute Gleichverteilung bei allein vier Generatoren.
\begin{figure}[ht]
  \centering
  \includegraphics[width=0.8\linewidth]{build/plot_01a.png}
  \caption{Pseudo Zufallsgeneratoren für vier Parametersätze.}%
  \label{fig:1}
\end{figure}

\subsection*{c)}
Es wird auf Korrelationen zwischen den jeweils aufeinanderfolgenden Zufallszahlen geprüft, indem sie gegeneinander in einem 2d-Histogramm aufgetragen werden (Abbildung 2).
Es werden Hyperebenen sichtbar, dies bedeutet, dass die Zufallszahlen korreliert sind (Parametersätze 1 und 2).
\begin{figure}[ht]
  \centering
  \includegraphics[width=0.8\linewidth]{build/plot_01b.png}
  \caption{Korrelationen der Zufallszahlen der vier Generatoren.}%
  \label{fig:2}
\end{figure}

\subsection*{d)}
Das gleiche Vorgehen wie in b) und c) wird für den XOR-Shift-Algorithmus angewendet.
Es ist zu erkennen, dass das zweite Tupel $(11, 4, 7)$ keine Gleichverteilung erzeugt und aufeinanderfolgende Zufallszahlen miteinander korrelieren (Abbildung 3).
\begin{figure}[ht]
  \centering
  \includegraphics[width=0.8\linewidth]{build/plot_01d.png}
  \caption{Histogramme und Korrelationen der XOR-Shift Algorithmen.}%
  \label{fig:3}
\end{figure}
\subsection*{e)}
Es wird eine Gridsearch für die Parameter $(11, b, c)$ durchgeführt, um herauszufinden, welche eine niedrige Rekursionsrate (= hohe Rekursionslänge) haben.
Dies gilt für die gelben Punkte in Abbildung 4.
Ein Beispiel für die Parameter eines guten XOR-Shift Algorithmus ist $(11, 1, 7)$, wie er auch in d) überprüft wurde.
\begin{figure}[ht]
  \centering
  \includegraphics[width=0.8\linewidth]{build/plot_01e.png}
  \caption{Gridsearch der besten Parameter des XOR-Shift Algorithmus für 16-Bit-Integer.}%
  \label{fig:4}
\end{figure}

\section*{Aufgabe 2}
\subsection*{a)}
Es wird die folgende Transformationsmethode verwendet:
\begin{align*}
  f(x) &= \cos(x) \\
  F(x) &= \int f(x)dx = \sin(x) := y \\
  \intertext{$y$ gleichverteilt $\in \left[0, 1\right[$}
  \Rightarrow x &= \arcsin(y) \\
  \intertext{$y$ cosinusverteilt $\in \left[\arcsin(0) = 0, \arcsin(1) = \frac{\pi}{2}\right[$.}
\end{align*}
Es werden in Abbildung 5 die Histogramme der Verteilungen (i), (ii) aufgetragen zwischen 0 und $\frac{\pi}{2}$,
da dies einen Vergleich zulässt.
Ausserdem ist die Differenz der Histogramme zur analytischen Funktion dargestellt, um die beiden Intervalle zu vergleichen.
Man erkennt, dass die Verteilungen gleich genau sind.
Der Unterschied zwischen den Eingangsintervallen ergibt den Unterschied in den transformierten Intervallen (nicht dargestellt, wegen Normierung):
\begin{align*}
  \left[0, 1\right[ &\rightarrow \left[0, \frac{\pi}{2}\right[ \\
  \left[-1, 1\right[ &\rightarrow \left[-\frac{\pi}{2}, \frac{\pi}{2}\right[
\end{align*}
\begin{figure}[ht]
  \centering
  \includegraphics[width=0.8\linewidth]{build/plot_02a.png}
  \caption{Histogramme nach $\cos(x)$ und Abweichung von (i) und (ii) von der analytischen Funktion.}%
  \label{fig:5}
\end{figure}
\subsection*{b)}
Es werden gleichverteilte Zufallszahlen $U_{1,2}$ in $\left[0, 1\right[$ erzeugt und nach 
\begin{align*}
  Z_{1} &= \mu + {\sqrt {-2\ln U_{1}}}\cos(2\pi U_{2}) \cdot \sigma\\
  Z_{2} &= \mu + {\sqrt {-2\ln U_{1}}}\sin(2\pi U_{2}) \cdot \sigma
\end{align*}
(Box-Muller Algorithmus) transformiert.
Das Histogramm der Zahlen sowie die analytische Form der Gaußkurve sind in Abbildung 6 aufgetragen.
\begin{figure}[ht]
  \centering
  \includegraphics[width=0.8\linewidth]{build/plot_02b.png}
  \caption{Box-Muller Algorithmus.}%
  \label{fig:6}
\end{figure}
\subsection*{c)}
Es wird eine Exponentialverteilung aus gleichverteilten Zufallszahlen in $\left[0, 1\right[$ erzeugt.
Dann wird für jede dieser Zahlen ein $y$-Wert aus einer Gleichverteilung zwischen $\left[0, k \cdot \exp(-x)\right[$ gezogen.
Ist dieser Wert kleiner als die Gaußfunktion an dieser Stelle, wird der Wert behalten.
So wird eine Gaußverteilung mit dem Neumannschen Rückweisungsverfahren generiert.
Es wird $k = 0.7$ gewählt.
In Abbildung 7 sind die normierten Histogramme der Verteilungen, die analytischen Funktionen und einige der gezogenen Zahlen abgebildet.
\begin{figure}[ht]
  \centering
  \includegraphics[width=0.8\linewidth]{build/plot_02c.png}
  \caption{Neumannsches Rückweisungsverfahren.}%
  \label{fig:7}
\end{figure}

\end{document}
