\documentclass{scrartcl}
\usepackage{graphicx}
\usepackage{amsmath}
\usepackage{subcaption}
\usepackage[
  section, % Floats innerhalb der Section halten
  below,   % unterhalb der Section aber auf der selben Seite ist ok
]{placeins}

\title{Übungsblatt 03.1}
\author{%
  Noah Biederbeck, Maximilian Sackel, Jan Spinne
}
\date{Abgabe: 11. Mai 2018}



\begin{document}
\maketitle

Drei Stimmen fuer Klausur.

\section*{Aufgabe 1}
\subsection*{(a)}
\begin{align*}
  s &= \pm 1 \\
  \mathcal{H} &= -sH \\
  Z &= \sum_i e^{-\beta \mathcal{H}(i)} \\
  \beta &= \frac{1}{kT} \\
  kT &= 1 \Rightarrow \beta = 1 \\
  Z &= e^{-\mathcal{H}(1)} + e^{ \mathcal{H}(2) } \\
  \left<O\right> &= \sum_{1, 2} \frac{1}{Z} e^{-\beta \mathcal{H} (i)} O(i) \\
  \intertext{$i$ jeder mgl Zustand}
                 &= \frac{1}{Z} \left( e^{-H} O(1) + e^H  O(2) \right)\\
                 &= \frac{1}{e^{-H} + e^H} \left( e^{-H} O(1) + e^H  O(2) \right) \\
  \intertext{Observable $O$ $\rightarrow$ Magnetisierung $m = s$}
  \left<m\right> &= \frac{1}{e^{-H} + e^H} \left( e^{-H} \cdot 1 + e^H \cdot (-1) \right)\\
                 &= \frac{e^{-H} - e^H}{e^{-H} + e^H} \\
                 &= \frac{-2 \sinh{H}}{2 \cosh{H}} \\
                 &= -\tanh{H}
\end{align*}
\end{document}
