\documentclass{scrartcl}
\usepackage{graphicx}
\usepackage{amsmath}
\usepackage{subcaption}
\usepackage[
section, % Floats innerhalb der Section halten
below,   % unterhalb der Section aber auf der selben Seite ist ok
]{placeins}

\title{Übungsblatt 05}
\author{%
		Noah Biederbeck, Maximilian Sackel, Jan Spinne
}
\date{Abgabe: 01. Juni 2018}



\begin{document}
\maketitle

\section*{Aufgabe 1: Diagonalisierung per Hand}
Matrix aus der Aufgabe:
\begin{equation}
  \label{eq:M}
  \mathbf{M} = \left(\begin{matrix}
      1 & 1 & 1 & 1 & 1 \\
      1 & 2 & 1 & 1 & 1 \\
      1 & 1 & 1 & 3 & 1 \\
      1 & 1 & 1 & 1 & 4 \\
      1 & 1 & 1 & 1 & 1
  \end{matrix}\right)
\end{equation}
Matrix nach dem Householder-Algorithmus:
\begin{equation}
  \label{eq:H}
  \mathbf{H} = \left(\begin{matrix}
    \input{build/householder_final.txt}
\end{matrix}\right)
\end{equation}
\begin{figure}[ht]
  \centering
  \includegraphics[width=0.8\linewidth]{build/eigenvalues.png}
  \caption{Eigenwerte der Matrix $\mathbf{M}$~\eqref{eq:M}, berechnet durch \texttt{Eigen::EigenSolver()} (blau) und selbstimplementiert (orange).}%
  \label{fig:eigenvalues}
\end{figure}
\end{document}
